\documentclass{beamer}
\usepackage[utf8]{inputenc}
\usepackage{hyperref}
\usepackage{multicol}
\usepackage{hyperref}

\inputencoding{utf8}

\mode<presentation> {
    \usetheme{Madrid}
}

\usepackage{graphicx}
\usepackage{booktabs}

\title[Introducciòn]{Introducci\'on al curso Inform\'atica 2}
\author{Ernesto Rodriguez}
\institute{
    Universidad del Itsmo \\
    \medskip \textit{erodriguez@unis.edu.gt}
}

\date[\today]{}

\begin{document}

\begin{frame}
\titlepage
\end{frame}

\begin{frame}
    \frametitle{Objetivos del curso}
    \begin{itemize}
        \item{Conocer de forma generalizada los diferentes campos dentro de
        las Ciencias de la Computaci\'on.}
        \item{Familiarizarse con las herramientas m\'as importantes de
        las ciencias de la computaci\'on.
        \begin{itemize}
            \item{Metodos formales}
            \item{L\'ogica y matematica fundamental}
            \item{Modelos de computaci\'onales}
            \item{Programaci\'on y tipos}
            \item{Control de versiones}
        \end{itemize}
        }
        \item{Decidir si las Ciencias de la Computaci\'on es
        la carrera correcta para usted.}
    \end{itemize}

    \small{Basado en \emph{General Computer Science I \& II} impartido
    por \href{https://kwarc.info/people/mkohlhase/}{Prof. Dr. Michael Kohlhase}}

\end{frame}

\begin{frame}
\frametitle{Acerca de Ernesto Rodriguez}
\begin{itemize}
    \item{Licenciatura de Jacobs University Bremen,
    enfocado en Machine Learning.}
    \item{Maestria de Utrecht University, enfocado
    en Lenguajes de Programaci\'on.}
    \item{Trabaje en varias empresas, incluyendo Microsoft.}
    \item{Dos publicaciones academicas.}
    \item{Varios proyectos open source.}
    \item{Intereses: Hackathons Cryptomonedas, Programaci\'on funcional,
    Machine Learning, Computaci\'on Teorica.}
\end{itemize}
\end{frame}

\begin{frame}
    \frametitle{Formato del curso}
    \begin{itemize}
        \item{Una hoja de trabajo (casi) semanal.}
        \item{Se utilizara Git y Github para entregar trabajos.}
        \item{Todo trabajo escrito se realizara con Latex.}
        \item{Tres ex\'amenes parciales t\'eoricos.}
        \item{Ex\'amen final.}
        \item{No es un curso de programaci\'on, pero
        se utilizara la porgramaci\'on para entender los temas.}
    \end{itemize}
\end{frame}

\begin{frame}
\frametitle{Herramientas y Recursos}
\begin{multicols*}{2}
    {\bf Herramientas} \\
\begin{itemize}
    \item \href{https://code.visualstudio.com/}{Visual Studio Code}
    \item \href{https://git-scm.com/}{Git}
    \item \href{https://github.com/}{Github}
    \item \href{https://www.latex-project.org/}{Latex}
    \item \href{https://nodejs.org/en/}{Node.js}
    \item \href{http://elm-lang.org/}{Elm}
    \item \href{https://manjaro.org/}{Manjaro Linux\\
    \tiny{No importa a que campo de la computaci\'on se dediquen,
        Linux siempre estara ahi.}}
\end{itemize}
\columnbreak
{\bf Recursos}
\begin{itemize}
    \item{Repositorio del curso \cite{Repositorio}}
    \item{General Computer Science \cite{GenCS}}
    \item Latex Wiki \cite{Latex}
    \item Git tutorial \cite{GitTutorial}
    \item .Net Core Guide \cite{DotNetGuide}
\end{itemize}
\end{multicols*}
\end{frame}

\begin{frame}

\frametitle{Reglas}

\begin{itemize}
\item{La asistencia a clase es requisito universitario, pero
somos adultos y si preferimos aprender por su cuenta, lo acepto.
Sin embargo, recomiendo que asistan a clase.}
\item{Durante clase, respetar al profesor y a sus compa\~neros.
No interrumpir ni burlarse de otras personas.}
\item{Ser respetuoso y constructivo a la hora de criticar.}
\item{No se tolera el plagio.
\\\small{Es m\'as dificil aprender a copiar que aprender a programar.}}
\item{Ser puntuales a la hora de entregar tareas.}
\end{itemize}
\end{frame}

\begin{frame}
\frametitle{Referencias}
\bibliography{../../../recursos/referencias}
\bibliographystyle{plain}
\end{frame}

\end{document}