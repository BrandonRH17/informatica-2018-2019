\documentclass{beamer}
\usepackage{amsmath}
\usepackage[utf8]{inputenc}
\usepackage{hyperref}
\usepackage{multicol}
\usepackage{hyperref}

\inputencoding{utf8}

\mode<presentation> {
    \usetheme{Madrid}
}

\usepackage{graphicx}
\usepackage{booktabs}

\title[Conjuntos]{Teoria de Conjuntos}
\author{Ernesto Rodriguez}
\institute{
    Universidad del Itsmo \\
    \medskip \textit{erodriguez@unis.edu.gt}
}

\date[\today]{}

\begin{document}

\begin{frame}
\frametitle{Conjuntos}
\begin{itemize}
\item{Los conjuntos sun una de las fundaciones de la matematica.}
\item{No se intentaran de definir, se intentara entenderlos.}
\item{Para entender un conjunto, hay que entender que elementos
son miembros y cuales no.}
\item{Los conjuntos se representan con:
\begin{itemize}
    \item{Listas de elementos: $\{a,b,c\}$}
    \item{Describir los elementos en base a una propiedad:
    $\{x|x \mathtt{tiene la propiedad} P\}$}
    \item{Mencionando la pertenencia: $a\in S$ \'o $b\notin S$}
    \item{{\bf Axioma}: Todos los elementos que podemos escribir
    existen.}
    \item{Es importante diferenciar objetos de su representaci\'on.}
\end{itemize}
}
\end{itemize}
\end{frame}

\begin{frame}
    \frametitle{Relaciones entre Conjuntos}
    \begin{itemize}
        \item{{\bf igualdad}: $(A\equiv B):\Leftrightarrow (
            \forall x.x\in A\Leftrightarrow x \in B)$}
        \item{{\bf subconjunto}: $(A\subseteq B:\Leftrightarrow
        (\forall x.x\in A\Rightarrow x\in B)$}
        \item{{\bf subconjunto propio}: $(A\subset B):\Leftrightarrow 
        (A\subseteq B)\wedge (A\not\equiv B)$}
        \item{{\bf superconjunto}: $(A\sqsupseteq B:\Leftrightarrow
        (\forall x.x\in B\Rightarrow x\in A$}
        \item{{\bf subconjunto propio}: $(A\supset B):\Leftrightarrow
        (A\supseteq B)\wedge(A\not\equiv B)$}
    \end{itemize}
\end{frame}

\begin{frame}
    \frametitle{Operaciones con Conjuntos}
    \begin{itemize}
        \item{{\bf union}: $A\cup B:=\{x|x\in A\vee x\in B\}$}
        \item{{\bf union sobre una colecci\'on}: Sea $I$ un
        \emph{conjunto} y $S_i$ una familia de conjuntos
        con indices $I$, entonces $\cup_{i\in I}S_i := \{
        x|\exists i\in I x.\in S_i \}$}
        \item{{\bf intersecci\'on}: $A\cap B := \{x|x\in A\wedge x\in B \}$}
        \item{{\bf intersecci\'on de una colecci\'on}: Sea $I$ un
        \emph{conjunto} y $S_i$, una familia de conjuntos con
        indices pertenecientes a $I$, entonces $\cap_{i\in I}S_i:=
        \{x|\forall i\in I.x\in S_i\}$}
        \item{{\bf diferencia de conjuntos}: $A\backslash B:=\{x
        |x\in A\wedge x\not\in B\}$}
        \item{{\bf conjunto de subconjuntos}: $\mathcal{P}(A)$ o
        $\{\}^{A}:=\{S|S\subseteq A\}$}
        \item{{\bf el conjunto vacio}: $\forall x.x\not\in \varnothing$}
    \end{itemize}
\end{frame}

\begin{frame}
\frametitle{Operaciones en Conjuntos (cont.)}
\begin{itemize}
    \item{{\bf Producto cartesiano}: $A\times B:=\{\langle a,b\rangle |
    a\in A \wedge b\in B\}$. $\langle a,b \rangle$ es una \emph{pareja}}
    \item{{\bf Producto cartesiano orden $n$}: $A_1\times\ldots\times A_n := \{
        \langle a_1,\ldots,a_n\rangle | \forall i.1\leq i \leq n \Rightarrow
        a_i\in A_i$}
    \item{{\bf Espacio cartesiano $n$-dimensional}: $A^n := \{
        \langle a_1,\ldots,a_n \rangle | 1\leq i\leq n \Rightarrow
        a_i\in A \}$, $\langle a_1,\ldots,a_n\rangle$ es un \emph{vector}}
        \item{{\bf Definicion:} se escribe $\cup_{i=1}^{n}S_i$ para
        el conjunto $\cup_{i\in\{i | 1\leq i \leq n \}}$}
        \item{{\bf Definicion:} se escribe $\cap_{i=1}^{n}S_i$ para
        el conjunto $\cap_{i\in\{i | 1\leq i \leq n \}}$}
\end{itemize}
\end{frame}

\begin{frame}
\frametitle{Tama\~nos de Conjuntos}
\begin{itemize}
    \item{{\bf Definici\'on}: El \emph{tama\~no} (\#$A$) de un
    conjunto $A$ es el numero de elementos en el conjunto.}
    \item{{\bf Conjetura}:
    \begin{itemize}
        \item{$\#\{a,b,c\}=3$}
        \item{$\#\mathbb{N}=\infty$}
        \item{$\#(A\cup B)\leq \#A+\#B$}
        \item{$\#(A\cap B)\leq \mathtt{min}(\#A,\#B)$}
        \item{$\#(A\times B)=\#A*\#B$}
    \end{itemize}
    }
    \item{¿Como demostramos estas propiedades?}
    \item{A todo esto, ¿Que significa ``numero de objetos''?}
    \item{{\bf Idea}: Es necesario asignarle a cada elemento
    de un conjunto un numero natural unario}
    \item{¿Como asociamos elementos de un conjunto a elementos
    de otros conjuntos?}
\end{itemize}
\end{frame}

\begin{frame}
\frametitle{Los conjuntos son impresionantes}
\begin{itemize}
    \item{Los conjuntos parecen simples, pero son \emph{poderosos}}
    \item{Hay conjuntos muy grandes. e.j. ``El conjunto $\mathcal{S}$ de todos los conjuntos'':
    \begin{itemize}
        \item{Contiene el conjunto $\varnothing$}
        \item{Para todo objeto $O$, tenemos que $\{O\}, \{\{O\}\}$, $\{O,\{O\}\},\ldots\in\mathcal{S}$}
        \item{Contiene todas las uniones, intersecciones y conjuntos de conjuntos}
        \item{Se contiene a si mismo $\mathcal{S}\in\mathcal{S}$}
    \end{itemize}
    }
\end{itemize}
\end{frame}

\begin{frame}
    \frametitle{El conjunto $\mathcal{S}$}
    \begin{itemize}
        \item{{\bf Propuesta}: El conjunto $\mathcal{S'}$ de todos los
        conjuntos que no se contienen a si mismo}
        \item{{\bf Pregunta}: ¿$\mathcal{S'}\in\mathcal{S'}$?
        \begin{itemize}
            \item{Supongamos que si, entonces $\mathcal{S'}\not\in\mathcal{S'}$
            ya que $\mathcal{S'}$ solo contiene objetos que no se contienen a si mismos}
            \item{Supongamos que no, entonces $\mathcal{S'}\in\mathcal{S'}$ ya que
            $\mathcal{S'}$ contiene a los conjuntos que no se contienen a si mismo}
        \end{itemize}
        }
        \item{De cualquier forma, tanto $\mathcal{S'}\in\mathcal{S'}$
        y $\mathcal{S'}\not\in\mathcal{S'}$ son ciertos, lo cual es una
        contradicci\'on}
    \end{itemize}
\end{frame}

\begin{frame}
    \frametitle{Jerga Matematica}
    \begin{itemize}
        \item{¿Nos proteje la \emph{jerga matematica}?}
        \item{{\bf No}: $\mathcal{S'}:=\{m|m\not\in m\}$}
        \item{La \emph{jerga matematica} nos permite construir
        enunciados contradictorios que son ``gramaticalmente''
        validos}
        \item{Hay que tener cuidado al construir conjuntos!}
    \end{itemize}
\end{frame}

\end{document}