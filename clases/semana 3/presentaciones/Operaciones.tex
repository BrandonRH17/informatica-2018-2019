\documentclass{beamer}
\usepackage{amsmath}
\usepackage[utf8]{inputenc}
\usepackage{hyperref}
\usepackage{multicol}
\usepackage{hyperref}

\inputencoding{utf8}

\mode<presentation> {
    \usetheme{Madrid}
}

\usepackage{graphicx}
\usepackage{booktabs}

\title[Operaciones]{Operaciones con Numeros Naturales Unarios}
\author{Ernesto Rodriguez}
\institute{
    Universidad del Itsmo \\
    \medskip \textit{erodriguez@unis.edu.gt}
}

\date[\today]{}

\begin{document}

\begin{frame}
\titlepage
\end{frame}

\begin{frame}
    \frametitle{¿Que podemos hacer con los Numeros Naturales Unarios?}
    \begin{itemize}
        \item{De momento, no mucho...}
        \item{¿Que necesitamos hacer con los numeros naturales unarios?}
        \item{¿Es posible expresar estas necesidades mediante un grupo
        peque\~no de operaciones (funciones)?}
        \item{¿Como definimos estas operaciones?}
        \item{¿Es posible construir algoritmos que lleven a cabo estas operaciones?}
    \end{itemize}
\end{frame}

\begin{frame}
\frametitle{Suma}
\begin{itemize}
    \item{{\bf Definici\'on: }Definimos la suma $\oplus$ para cualquier numero
    natural unario $n$ y $m$ como:
    \[
        n\oplus m \left\{
        \begin{array}{l l}
            m & \mbox{si } n=o \\
            n & \mbox{si } m=o \\
            s(i\oplus m) & \mbox{si } n=s(i) \\
        \end{array}
        \right.
    \]
    }
    \item{¿Es una definici\'on buena?}
    \item{¿Pueden estas funciones definirse mediante algoritmos?}
    \item{A todo esto, ¿Que es una funci\'on?}
    \item{¿Ejemplos?}
\end{itemize}
\end{frame}

\begin{frame}
\frametitle{Substituci\'on}
\begin{itemize}
    \item{{\bf Definici\'on: }A una agrupaci\'on de numeros naturales unarios y
    operaciones se le conoce como expresi\'on y se abreviaran mediante variables como $\xi$}
    \item{{\bf Ejemplo: }$m\oplus n \oplus l$ es una expresi\'on}
    \item{{\bf Definici\'on: }Dado que $n=m$ y una expresi\'on arbitraria $\xi$,
    es permitido substituir todas las ocurrencias de $n$ por $m$ en $\xi$.}
    \item{{\bf Definici\'on: }La expresi\'on $\xi$ donde $m$ es sustituido por $n$
    se denota como $\xi[m \slash n]$}
    \item{{\bf Ejemplo: }Si $m=s(s(o))$ y $\xi=m\oplus x$, entonces $\xi[m \slash s(s(o))]=s(s(o))\oplus n$}
\end{itemize}
\end{frame}

\begin{frame}
    \frametitle{La adici\'on es asociativa}
    {\bf Teorema (asociatividad):} Para todo numero natural unario $n,m,l$ se cumple:
    $n\oplus(m\oplus l)=(n\oplus m)\oplus l$
    \\[1cm]
    {\bf Demostraci\'on: mediante inducci\'on en $l$}
    \begin{itemize}
        \item{{\bf Caso base: $l=o$}
        \begin{itemize}
            \item{$n\oplus (m\oplus o)=n \oplus m = (n \oplus m)\oplus o$}
        \end{itemize}
        }
        \item{{\bf Caso inductivo: $l=s(i)$}
        \begin{enumerate}
            \item{{\bf Hipotesis inductiva: }Asumimos que $m\oplus (n\oplus i)=(m\oplus n)\oplus i$}
            \item{Sabemos que $n\oplus(m\oplus s(i))=n\oplus s(m\oplus i)=s(n\oplus(m\oplus i))$}
            \item{Sabemos que $s(n\oplus(m\oplus i))=s((n\oplus m)\oplus i)$ por la hipotesis inductiva}
            \item{Concluimos que $s((n\oplus m)\oplus i)=(n\oplus m)\oplus s(i)$}
        \end{enumerate}
        }
    \end{itemize}
\end{frame}

\end{document}